\documentclass{beamer}
\usepackage{listings}
\usepackage{comment}

\begin{document}


\begin{frame}[fragile]
\frametitle{Functions and functional programming}
Python is not a functional programming language, but
it has a lot of features taken from functional
programming languages:
\begin{itemize}
  \item closures
  \item high order functions and decorators
  \item generators
  \item corutines
  \item list comprehensions
\end{itemize}

\end{frame}


\begin{frame}[fragile]
\frametitle{But First... Let's define a function}

\begin{lstlisting}[language=python]
def add(x, y):
    return x + y
\end{lstlisting}
\pause
\begin{lstlisting}[language=python]
s = add(x, y)
print s
\end{lstlisting}

\pause
\vspace{5 mm}
That's it!!


\pause
\vspace{5 mm}
Not actually... There are a couple of things to notice:
\begin{itemize}
  \item python is dynamic
  \item parameters don't have a specified type
  \item neither do we specify the return type
\end{itemize}
\begin{comment}
  Don't forget to add that any types that support the + 
  operator are acceptable
\end{comment}

\end{frame}


\begin{frame}[fragile]
\frametitle{Evaluation Strategy}

\begin{itemize}
  \item parameters are just names that point to objects
  \item if you pass an immutable object, it looks as if it was passed by value
\pause
\begin{lstlisting}[language=python]
def increment_value(x):
    x = x + 1
    print x  # Outputs: 4

a = 3
increment_value(y)
print a  # Outputs: 3 
\end{lstlisting}
  \item Oups... It didn't actually work...
\end{itemize}
\end{frame}


\begin{frame}[fragile]
\frametitle{Evaluation Strategy}

\begin{itemize}
  \item parameters are just names that point to objects
  \item if you pass a mutable object and that object is modified, the changes are going
    to be visible in the caller
\end{itemize}
\pause
\begin{lstlisting}[language=python]
def increment(values):
    for i, v in enumerate(values):
       values[i] = v + 1

a = [1, 2, 3]
increment(a)
print a  # Outputs: [2, 3, 4]
\end{lstlisting}
\end{frame}

\begin{frame}[fragile]
\frametitle{Evaluation Stragegy}

Functions that mutate their input arguments are:

\begin{itemize}
  \item said to have {\bf side effects}
  \item are best avoided as they might lead to to subtle bugs
  \item are needed for doing {\bf in-place} changes to large or expensive objects
\end{itemize}

\end{frame}


\begin{frame}[fragile]
\frametitle{Evaluation Strategy}

\begin{itemize}
  \item this is named {\bf call by sharing}
  \item it's the same as in languages such as Java or Ruby
  \item though Java people name it pass-by-value
  \item while Ruby people name it pass-by-reference
\end{itemize}
\end{frame}


\begin{frame}[fragile]
\frametitle{Exercises}

\begin{enumerate}
  \item write a function that takes a list of integers and returns the number of even numbers containd in the list
  \item write a function that takes a list of integers and returns a new list containing the even numbers from the list
  \item write a function that takes a list of integers and {\bf in-place} removes the odd elements
\end{enumerate}

\end{frame}


\begin{frame}[fragile]
\frametitle{Exercise 3. Take 1}

\begin{lstlisting}[language=python]
def remove_odd(values):
    for val in values:
        if val % 2 != 0:
            values.remove(val)

a = [1, 1, 1, 2, 4]
remove_odd(a)  # list will be [1, 2, 4]
\end{lstlisting}
\vspace{5 mm}
Never add/remove elements while iterating!!
\end{frame}

\begin{frame}[fragile]
\frametitle{Exercise 3. Take 2}
\begin{lstlisting}[language=python]
import copy

def remove_odd(values):
    for val in copy.copy(values):
        if val % 2 != 0:
            values.remove(val)

a = [1, 1, 1, 2]
remove_odd(a)  # list will correctly be [2, 4]
\end{lstlisting}
\vspace{5 mm}
This works, but the algorithm is O(n^2)
\end{frame}


\begin{frame}[fragile]
\frametitle{Exercise 3. Take 3}
\begin{lstlisting}[language=python]
def remove_odd(values):
    values[:] = [v for v in values if v % 2 == 0]

a = [1, 1, 1, 2]
remove_odd(a)  # list will correctly be [2, 4]
\end{lstlisting}
\end{frame}


\begin{frame}[fragile]
\frametitle{Default parameter values}

\begin{lstlisting}[language=python]
def increment(x, inc=1):
   return x + inc

a = 3
increment(a)  # returns: 4
increment(a, 2)  # returns: 6
\end{lstlisting}
\end{frame}

\begin{frame}[fragile]
\frametitle{Default parameter values}

\begin{itemize}
  \item you can't have a non-default parameter following a default one. That raises {\bf SyntaxError}
  \item default parameter values are are assigned at function definition and never change
\end{itemize}

\begin{lstlisting}[language=python]
default = 1
def foo(x=defualt)
    print x

default = 2
foo()  # Outputs: 1
\end{lstlisting}
\end{frame}

\begin{frame}[fragile]
\frametitle{Keyword arguments}

\begin{lstlisting}[language=python]
def make_symlink(target, link_name):
    do stuff

make_symlink(target='/foo', link_name='/bar')
\end{lstlisting}
\pause
\begin{lstlisting}[language=python]
make_symlink(link_name='/bar', target='/foo')
\end{lstlisting}
\pause
\begin{lstlisting}[language=python]
make_symlink('/foo', link_name='/bar')
\end{lstlisting}
\pause
\begin{lstlisting}[language=python]
make_symlink(target='foo', '/bar')  # SyntaxError !!
\end{lstlisting}
\end{frame}


\begin{frame}[fragile]
\frametitle{Varargs functions}

\begin{lstlisting}[language=python]

def make_window(parent, *args, **kwargs):
    print container
    print args     
    print kwargs   
                   
                   

make_window(1, 2, 3, 4, 5,
   color='red',
   modal=False,
   visible=True)

\end{lstlisting}
\end{frame}


\begin{frame}[fragile]
\frametitle{Varargs functions}

\begin{lstlisting}[language=python]

def make_window(parent, *args, **kwargs):
    print container  # Outputs: 1
    print args       # Outputs: (2, 3, 4, 5)
    print kwargs     # Outputs: {'color': 'red', 
                     #           'modal': False,
                     #           'visible': True}

make_window(1, 2, 3, 4, 5,
   color='red',
   modal=False,
   visible=True)

\end{lstlisting}
\end{frame}

\begin{frame}[fragile]
\frametitle{Variable scope}
Python uses function scope:
\begin{itemize}
 \item each time a function executes a new local namespace is created
 \item the local namespace contains parameters as well as variables defined inside the function
\end{itemize}
\vspace{5 mm}
When resolving variables
\begin{itemize}
  \item the local namespace is searched
  \item If no match is found, the global namespace is searched  
\end{itemize}

\end{frame}

\begin{frame}[fragile]
\frametitle{Variable scope}
\begin{lstlisting}[language=python]
var = 10
def foo():
    var = 21
foo()
print var  # Outputs: 10
\end{lstlisting}
\end{frame}

\begin{frame}[fragile]
\frametitle{Variable scope}
\begin{lstlisting}[language=python]
var = 10
def foo():
    global var
    var = 21
foo()
print var  # Outputs: 21
\end{lstlisting}
\end{frame}


\begin{frame}[fragile]
\frametitle{Nested functions}
\begin{lstlisting}[language=python]
def foo(x):
    def bar(y):
        
\end{lstlisting}
\end{frame}


\end{document}

