\documentclass{beamer}
\usepackage{listings}
\usepackage{comment}

\begin{document}


\begin{frame}[fragile]
\frametitle{Functions and functional programming}
Python is not a functional programming language, but
it has a lot of features taken from functional
programming languages:
\begin{itemize}
  \item closures
  \item high order functions and decorators
  \item generators
  \item corutines
  \item list comprehensions
\end{itemize}

\end{frame}


\begin{frame}[fragile]
\frametitle{But First... Let's define a function}

\begin{lstlisting}[language=python]
def add(x, y):
    return x + y
\end{lstlisting}
\pause
\begin{lstlisting}[language=python]
s = add(x, y)
print s
\end{lstlisting}

\pause
\vspace{5 mm}
That's it!!


\pause
\vspace{5 mm}
Not actually... There are a couple of things to notice:
\begin{itemize}
  \item python is dynamic
  \item parameters don't have a specified type
  \item neither do we specify the return type
\end{itemize}
\begin{comment}
  Don't forget to add that any types that support the + 
  operator are acceptable
\end{comment}

\end{frame}


\begin{frame}[fragile]
\frametitle{Evaluation Strategy}

\begin{itemize}
  \item parameters are just names that point to objects
  \item if you pass an immutable object, it looks as if it was passed by value
\pause
\begin{lstlisting}[language=python]
def increment_value(x):
    x = x + 1
    print x  # Outputs: 4

a = 3
increment_value(y)
print a  # Outputs: 3 
\end{lstlisting}
  \item Oups... It didn't actually work...
\end{itemize}
\end{frame}


\begin{frame}[fragile]
\frametitle{Evaluation Strategy}

\begin{itemize}
  \item parameters are just names that point to objects
  \item if you pass a mutable object and that object is modified, the changes are going
    to be visible in the caller
\end{itemize}
\pause
\begin{lstlisting}[language=python]
def increment(values):
    for i, v in enumerate(values):
       values[i] = v + 1

a = [1, 2, 3]
increment(a)
print a  # Outputs: [2, 3, 4]
\end{lstlisting}
\end{frame}

\begin{frame}[fragile]
\frametitle{Evaluation Strategy}

Functions that mutate their input arguments are:

\begin{itemize}
  \item said to have {\bf side effects}
  \item are best avoided as they might lead to to subtle bugs
  \item are needed for doing {\bf in-place} changes to large or expensive objects
\end{itemize}

\end{frame}


\begin{frame}[fragile]
\frametitle{Evaluation Strategy}

\begin{itemize}
  \item this is named {\bf call by sharing}
  \item it's the same as in languages such as Java or Ruby
  \item though Java people name it pass-by-value
  \item while Ruby people name it pass-by-reference
\end{itemize}
\end{frame}


\begin{frame}[fragile]
\frametitle{Exercises}

\begin{enumerate}
  \item write a function that takes a list of integers and returns the number of even numbers contained in the list
  \item write a function that takes a list of integers and returns a new list containing the even numbers from the list
  \item write a function that takes a list of integers and {\bf in-place} removes the odd elements
\end{enumerate}

\end{frame}


\begin{frame}[fragile]
\frametitle{Exercise 3. Take 1}

\begin{lstlisting}[language=python]
def remove_odd(values):
    for val in values:
        if val % 2 != 0:
            values.remove(val)

a = [1, 1, 1, 2, 4]
remove_odd(a)  # list will be [1, 2, 4]
\end{lstlisting}
\vspace{5 mm}
This is wrong!! Never add/remove elements while iterating!!
\end{frame}

\begin{frame}[fragile]
\frametitle{Exercise 3. Take 2}
\begin{lstlisting}[language=python]
import copy

def remove_odd(values):
    for val in copy.copy(values):
        if val % 2 != 0:
            values.remove(val)

a = [1, 1, 1, 2]
remove_odd(a)  # list will correctly be [2, 4]
\end{lstlisting}
\vspace{5 mm}
This works, but the algorithm is O(n^2)
\end{frame}


\begin{frame}[fragile]
\frametitle{Exercise 3. Take 3}
\begin{lstlisting}[language=python]
def remove_odd(values):
    values[:] = [v for v in values if v % 2 == 0]

a = [1, 1, 1, 2]
remove_odd(a)  # list will correctly be [2, 4]
\end{lstlisting}
\vspace{5 mm}
More about this when we talk about list comprehensions.
\end{frame}


\begin{frame}[fragile]
\frametitle{Default parameter values}

\begin{lstlisting}[language=python]
def increment(x, inc=1):
   return x + inc

a = 3
increment(a)  # returns: 4
increment(a, 2)  # returns: 6
\end{lstlisting}
\end{frame}

\begin{frame}[fragile]
\frametitle{Default parameter values}

\begin{itemize}
  \item you can't have a non-default parameter following a default one. That raises {\bf SyntaxError}
  \item default parameter values are are assigned at function definition and never change
\end{itemize}

\pause

\begin{lstlisting}[language=python]
default = 1
def foo(x=default)
    print x

default = 2
foo()  # Outputs: 1
\end{lstlisting}
\end{frame}

\begin{frame}[fragile]
\frametitle{Keyword arguments}

\begin{lstlisting}[language=python]
def make_symlink(target, link_name):
    do stuff

make_symlink(target='/foo', link_name='/bar')
\end{lstlisting}
\pause
\begin{lstlisting}[language=python]
make_symlink(link_name='/bar', target='/foo')
\end{lstlisting}
\pause
\begin{lstlisting}[language=python]
make_symlink('/foo', link_name='/bar')
\end{lstlisting}
\pause
\begin{lstlisting}[language=python]
make_symlink(target='foo', '/bar')  # SyntaxError !!
\end{lstlisting}
\end{frame}


\begin{frame}[fragile]
\frametitle{Varargs functions}

\begin{lstlisting}[language=python]

def make_window(parent, *args, **kwargs):
    print container
    print args     
    print kwargs   
                   
                   

make_window(1, 2, 3, 4, 5,
   color='red',
   modal=False,
   visible=True)

\end{lstlisting}
\end{frame}


\begin{frame}[fragile]
\frametitle{Varargs functions}

\begin{lstlisting}[language=python]

def make_window(parent, *args, **kwargs):
    print container  # Outputs: 1
    print args       # Outputs: (2, 3, 4, 5)
    print kwargs     # Outputs: {'color': 'red', 
                     #           'modal': False,
                     #           'visible': True}

make_window(1, 2, 3, 4, 5,
   color='red',
   modal=False,
   visible=True)

\end{lstlisting}
\end{frame}

\begin{frame}[fragile]
\frametitle{Variable scope}
Python uses function scope:
\begin{itemize}
 \item each time a function executes a new local namespace is created
 \item the local namespace contains parameters as well as variables defined inside the function
\end{itemize}
\vspace{5 mm}
When resolving variables
\begin{itemize}
  \item the local namespace is searched
  \item If no match is found, the global namespace is searched  
\end{itemize}

\end{frame}

\begin{frame}[fragile]
\frametitle{Variable scope}
\begin{lstlisting}[language=python]
var = 10
def foo():
    var = 21
foo()
print var  # Outputs: 10
\end{lstlisting}
\end{frame}

\begin{frame}[fragile]
\frametitle{Variable scope}
\begin{lstlisting}[language=python]
var = 10
def foo():
    global var
    var = 21
foo()
print var  # Outputs: 21
\end{lstlisting}
\end{frame}

\begin{frame}[fragile]
\frametitle{Nested functions}
\begin{lstlisting}[language=python]
def countdown(initial, msg):

    def show_msg():
        print '%s %d' % (msg, n)

    for n in xrange(initial, 0, -1):
        show_msg()

countdown(2, 'at:')
# Output:
# at:2
# at:1
# at:0
\end{lstlisting}
\end{frame}

\begin{frame}[fragile]
\frametitle{Functions as first class citizens}

What this means:
\begin{itemize}
  \item functions can be passed as parameters
  \item functions can be return values
\end{itemize}
\pause
\begin{lstlisting}[language=python]
def compare(x, y):
    return cmp(x.lower(), y.lower())

sorted(['B', 'c', 'a'], compare)
# Returns ['a', 'B', 'c']
\end{lstlisting}

\end{frame}


\begin{frame}[fragile]
\frametitle{Closures}
A closure is a function that is packaged together with the surrounding environment
\end{frame}


\begin{frame}[fragile]
\frametitle{Closure example}
Closures can be used for delayed evaluation
\vspace{5 mm}
\begin{lstlisting}[language=python]
from urllib import urlopen

def page(url):
    def get():
        return urlopen(url).read()
    return get

get_python = page('http://python.org')
get_jython = page('http://jython.org')

pydata = get_python()  # Fetches http://python.org
jydata = get_jython()  # Fetches http://jython.org
\end{lstlisting}
\end{frame}


\begin{frame}[fragile]
\frametitle{Closure}
Closure can be used for preserving state across function calls. Note: this only works in python 3 by using the {\bf nonlocal} keyword.

\vspace{5 mm}
\begin{lstlisting}[language=python]
def counter(initial_value):
    counter = initial_value
    def inc():
        nonlocal counter
        counter += 1
        return counter
    def dec():
        nonlocal counter
        counter -= 1
        return counter
    return inc, dec

inc, dec = counter(10)
inc()  # Returns: 11
inc()  # Returns: 12
dec()  # Returns: 11
\end{lstlisting}
\end{frame}


\begin{frame}[fragile]
\frametitle{Closure}
In python 2.7, if you want to re-assign a variable from the nesting function,
you have to wrap that in a list.
\vspace{5 mm}
\begin{lstlisting}[language=python]
def counter(initial_value):
    counter = [initial_value]
    def inc():
        counter[0] += 1
        return counter[0]
    def dec():
        counter[0] -= 1
        return counter[0]
    return inc, dec

inc, dec = counter(10)
print inc()  # Returns: 11
print inc()  # Returns: 12
print dec()  # Returns: 11

\end{lstlisting}
\end{frame}


\begin{frame}[fragile]
\frametitle{Alternative implementation}
The classic way of implementing the previous example
would have been by using a class

\vspace{5 mm}
\begin{lstlisting}[language=python]
class Counter(object):
    def __init__(self, initial_value):
        self.counter = initial_value

    def inc(self):
        self.counter += 1
        return self.counter

    def dec(self):
        self.counter -= 1
        return self.counter

counter = Counter(10)
counter.inc()  # Returns: 11
\end{lstlisting}
\end{frame}

\begin{frame}[fragile]
\frametitle{Exercise}
Implement a stack using closures.
\vspace{5 mm}
\begin{lstlisting}[language=python]
push, pop = stack()

push(1)
push(3)
pop()    # Returns: 3
push(4)
pop()    # Returns: 4
pop()    # Returns: 1
\end{lstlisting}
\end{frame}


\begin{frame}[fragile]
\frametitle{Solution}
\vspace{5 mm}
\begin{lstlisting}[language=python]
def stack():
    s = []
    def push(value):
        s.append(value)
    def pop():
        return s.pop()
    return push, pop

push, pop = stack()
push(1)
push(2)
print pop()
print pop()

\end{lstlisting}
\end{frame}


\begin{frame}[fragile]
\frametitle{High order functions}
High order functions are functions that do at least one of
\begin{itemize}
  \item take one or more functions as input
  \item return a function
\end{itemize}
\end{frame}

\begin{frame}[fragile]
\frametitle{High order function example 1}
\begin{lstlisting}[language=python]
def logging_wrapper(func):
    def wrapped():
        print 'entering'
        func()
        print 'exiting'
    return wrapped

def foo():
    print 'fooo'

logged_foo = logging_wrapper(foo)
logged_foo()
# entering
# fooo
# exiting
\end{lstlisting}
\end{frame}

\begin{frame}[fragile]
\frametitle{High order function example 2}
\begin{lstlisting}[language=python]
def logging_wrapper(func):
    def wrapped():
        print 'entering'
        func()
        print 'exiting'
    return wrapped

def foo():
    print 'fooo'

foo = logging_wrapper(foo)
foo()
# entering
# fooo
# exiting
\end{lstlisting}
\end{frame}


\begin{frame}[fragile]
\frametitle{Decorators (take 1)}
\begin{lstlisting}[language=python]
def logging_wrapper(func):
    def wrapped():
        print 'entering'
        func()
        print 'exiting'
    return wrapped

@logging_wrapper
def foo():
    print 'fooo'

foo()
# entering
# fooo
# exiting
\end{lstlisting}
\end{frame}

\begin{frame}[fragile]
\frametitle{Decorators (take 2)}
\begin{lstlisting}[language=python]
def logging_wrapper(func):
    def wrapped(*args, **kwargs):
        print 'entering'
        ret_val = func(*args, **kwargs)
        print 'exiting'
        return ret_val
    return wrapped

@logging_wrapper
def foo(msg):
    return 'fooo %s' % msg

print foo('bar')
# entering
# exiting
# fooo bar
\end{lstlisting}
\end{frame}

\begin{frame}[fragile]
\frametitle{Decorators exercise 1}
Write a 'timing' decorator that wraps a function and prints how long the function's execution takes

\vspace{15 mm}
{\bf Hint}: use the {\bf time} module
\vspace{5 mm}
\begin{lstlisting}[language=python]
import time

started_at = time.time()
# do stuff
print time.time() - started_at
\end{lstlisting}
\end{frame}

\begin{frame}[fragile]
\frametitle{Decorators exercise 2}

Write a decorator that keeps track of how many times decorated functions are being called

\vspace{5 mm}

\begin{lstlisting}[language=python]
@count_calls
def foo():
    print 'booo'

@count_calls
def bar():
    print 'boohoo yourself'

foo()
bar()
foo()
get_call_count(foo)  # Returns: 2
get_call_count(bar)  # Returns: 1
\end{lstlisting}

\end{frame}


\begin{frame}[fragile]
\frametitle{Generators}
\begin{itemize}
  \item a generator is a function that produces a sequence of values.
  \item the sequence can be then consumed with a {\bf for} loop or by explicitly
    calling {\bf next} on the returned generator object
\end{itemize}
\vspace{5 mm}
\pause
\begin{lstlisting}[language=python]
def my_range(first, last):
    i = first
    while i < last:
        yield i
        i += 1

for x in my_range(0, 3):
   print x

# Outputs: 0  1  2
\end{lstlisting}
\end{frame}

\begin{frame}[fragile]
\frametitle{Generators}
\begin{itemize}
  \item a generator is a function that produces a sequence of values.
  \item the sequence can be then consumed with a {\bf for} loop or by explicitly
    calling {\bf next} on the returned generator object
\end{itemize}
\vspace{5 mm}
\begin{lstlisting}[language=python]
def my_range(first, last):
    i = first
    while i < last:
        yield i
        i += 1

print sum(my_range(0, 3))  # Outputs: 4
\end{lstlisting}
\end{frame}

\begin{frame}[fragile]
\frametitle{Generators}
\begin{itemize}
  \item a generator is a function that produces a sequence of values.
  \item the sequence can be then consumed with a {\bf for} loop or by explicitly
    calling {\bf next} on the returned generator object
\end{itemize}
\vspace{5 mm}
\begin{lstlisting}[language=python]
def my_range(first, last):
    i = first
    while i < last:
        yield i
        i += 1

gen = my_range(0, 3)
print gen.next()  # Outputs 0
print gen.next()  # Outputs 1
print gen.next()  # Outputs 2
print gen.next()  # raised StopIteration !
\end{lstlisting}
\end{frame}

\begin{frame}[fragile]
\frametitle{Generators}
\vspace{5 mm}
\begin{lstlisting}[language=python]
def my_range(first, last):
    i = first
    while i < last:
        yield i
        i += 1

gen = my_range(0, 3)
while True:
   try:
      print gen.next()
   except StopIteration:
      break
\end{lstlisting}
\end{frame}

\begin{frame}[fragile]
\frametitle{Endless generators}
\vspace{5 mm}
\begin{lstlisting}[language=python]
import random

def random_generator():
    while True:
        yield random.random()

random_gen = random_generator()
for rand_nr in random_gen:
    print nr
    if rand_nr > 0.5:
        break
random_gen.close()
\end{lstlisting}
\end{frame}

\begin{frame}[fragile]
\frametitle{Exercises}
\begin{enumerate}
  \item Write a generator that takes an integer parameter and yields Fibonacci numbers
    smaller than the given parameter.
  \item Having a binary tree encoded as a tuple (label, left, right) write a generator that
    yields the labels in pre-order (root, left, right). Write both an iterative and recursive
    implementation.
\end{enumerate}
Example:
\begin{lstlisting}[language=python]
tree=  ('b',
          ('a',
             ('q', None, None),
              None),
          ('z',
              ('c', None, None),
              ('zz', None, None)))

for label in iterate(tree):
    print label
Output: b, a, q, z, c, zz
\end{lstlisting}
\end{frame}


\begin{frame}[fragile]
\frametitle{Iterative solution}
\begin{lstlisting}[language=python]
def iterator(bin_tree):
    node_stack = [bin_tree]
    while node_stack:
        node = node_stack.pop()
        if not node:
            continue
        label, left, right = node
        yield label
        node_stack.append(right)
        node_stack.append(left)
\end{lstlisting}
\end{frame}

\begin{frame}[fragile]
\frametitle{Recursive solution}
\begin{lstlisting}[language=python]
def iterator(bin_tree):
    if not bin_tree:
        return
    label, left, right = bin_tree
    yield label
    for label in iterator(left):
        yield label
    for label in iterator(right):
        yield label
\end{lstlisting}
\end{frame}

\begin{frame}[fragile]
\frametitle{Exercises pt. 2}
\begin{enumerate}
  \item write a generator that flattens a list that might have any level of nesting.
\end{enumerate}
Example:
\begin{lstlisting}[language=python]
for x in flatten([1, [2, 3], [[4, 5], 6, [[[7]]]):
    print x

Output: 1 2 3 4 5 6
\end{lstlisting}
\vspace{5 mm}
{\bf Hint}: use isinstance for checking if an object is a list or not
\end{frame}

\begin{frame}[fragile]
\frametitle{Exercise solution}
\begin{lstlisting}[language=python]
def flatten(lst):
    for item in lst:
        if isinstance(item, list):
            for sub_item in flatten(item):
                yield sub_item
        else:
            yield item

\end{lstlisting}
\end{frame}




\begin{frame}[fragile]
\frametitle{Piping generators}
\begin{lstlisting}[language=python]
def grep(lines, word):
    for line in lines:
        if word in line:
            yield line

f = open('/etc/passwd')
lines = grep(f, 'foo')
lines = grep(lines, 'bar')
for line in lines:
    print line
f.close()

# Equivalent: cat some_file | grep foo | grep bar
\end{lstlisting}
\end{frame}

\begin{frame}[fragile]
\frametitle{Piping generators}
\begin{lstlisting}[language=python]
def grep(lines, word):
    for line in lines:
        if word in line:
            yield line

f = open('/etc/passwd')
try:
    lines = grep(f, 'foo')
    lines = grep(lines, 'bar')
    for line in lines:
        print line
finally:
    f.close()
\end{lstlisting}
\end{frame}


\begin{frame}[fragile]
\frametitle{Piping generators}
\begin{lstlisting}[language=python]
def grep(lines, word):
    for line in lines:
        if word in line:
            yield line

with open('/etc/passwd') as f:
    lines = grep(f, 'foo')
    lines = grep(lines, 'bar')
    for line in lines:
        print line
\end{lstlisting}
\end{frame}


\begin{frame}[fragile]
\frametitle{Piping generators}
\begin{lstlisting}[language=python]
def grep(lines, word):
    for line in lines:
        if word in line:
            yield line

for ln in grep(grep(open('passwd'), 'foo'), 'bar'):
    print ln
\end{lstlisting}

\end{frame}

\begin{frame}[fragile]
\frametitle{List comprehensions}
\begin{lstlisting}[language=python]
nums = [1, 2, 3, 4, 5]
times_two = [x * 2 for x in nums]

print times_two
# Outputs:  [2, 4, 6, 8, 10]
\end{lstlisting}
\end{frame}

\begin{frame}[fragile]
\frametitle{List comprehensions}
\begin{lstlisting}[language=python]
nums = [1, 2, 3, 4, 5]
times_two = [x * 2 for x in nums if x % 2 == 0]

print times_two
# Outputs:  [4, 8]
\end{lstlisting}
\end{frame}

\begin{frame}[fragile]
\frametitle{Exercise}
\begin{itemize}
  \item write a list comprehension statement that given a list of string 
    excludes the ones longer than 5 characters
    and makes the shorter ones uppercase
\end{itemize}
\end{frame}

\begin{frame}[fragile]
\frametitle{List comprehensions}
\begin{lstlisting}[language=python]
sentences = ['mama are mere', 'tata are pere']
words = []
for sentence in sentences:
    for w in sentence.split():
        words.append(w.upper())

print words
# Outputs: ['MAMA', 'ARE', 'MERE', 'TATA',
            'ARE', 'PERE']

\end{lstlisting}
\end{frame}

\begin{frame}[fragile]
\frametitle{List comprehensions}
\begin{lstlisting}[language=python]
sentences = ['mama are mere', 'tata are pere']

words = [w.upper() for sentence in sentences
                   for w in sentence.split()]

print words
# Outputs: ['MAMA', 'ARE', 'MERE', 'TATA',
            'ARE', 'PERE']
\end{lstlisting}
\end{frame}


\begin{frame}[fragile]
\frametitle{Exercises}
\begin{itemize}
  \item Write a list comprehension statement that flattens a list of lists but
    skips nested lists that have a single element
\end{itemize}

\begin{lstlisting}[language=python]
Example:

[[1, 2], [3], [4, 5]] should be
transformed to [1, 2, 4, 5]
\end{lstlisting}
\end{frame}


\begin{frame}[fragile]
\frametitle{List comprehensions}
List comprehensions can be used for building generators as well
\begin{lstlisting}[language=python]
sentences = ['mama are mere', 'tata are pere']

words = (w.upper() for sentence in sentences
                   for w in sentence.split())

for word in words:
    print words
\end{lstlisting}
\end{frame}


\begin{frame}[fragile]
\frametitle{In-place list processing}
\begin{lstlisting}[language=python]
numbers = [1, 2, 3, 4, 5]
numbers[:] = [x * x for x in numbers if x % 2 == 0]

print numbers
# Outputs:  [4, 16]
\end{lstlisting}
\end{frame}

\begin{frame}[fragile]
\frametitle{Using list comprehensions for initializing container objects}
\begin{lstlisting}[language=python]
words = ['foo', 'barbaz', 'oups']
d = dict((w, len(w)) for w in words)

print d  # Outputs: {'foo: 3, 'barbaz': 6, 'oups': 4}
\end{lstlisting}
\end{frame}


\begin{frame}[fragile]
\frametitle{Lambdas}
\begin{itemize}
  \item lambdas are anonymous functions
  \item can have a single expression
  \item use-full for short callbacks
\end{itemize}
\begin{lstlisting}[language=python]
lambda x, y: x + y
\end{lstlisting}
\end{frame}


\begin{frame}[fragile]
\frametitle{Using lambdas with filter}
\begin{lstlisting}[language=python]
numbers = [-7, 3, 4, -8, 9]

positive_nums = [x for x in numbers if x >= 0]

# is the equivalent of
positive_nums = filter(lambda x: x >= 0, numbers)
\end{lstlisting}
\end{frame}

\begin{frame}[fragile]
\frametitle{Using lambdas with map}
\begin{lstlisting}[language=python]
numbers = [-7, 3, 4, -8, 9]

squares = [x * x for x in numbers]

# is the equivalent of
squares = map(lambda x: x * x, numbers)

\end{lstlisting}
\end{frame}

\begin{frame}[fragile]
\frametitle{Using lambdas with reduce}
\begin{lstlisting}[language=python]
numbers = [1, 2, 3, 4]

print reduce(lambda x, y: x + y, numbers)
# Outputs: 10

\end{lstlisting}
\end{frame}


\begin{frame}[fragile]
\frametitle{Exercise}

Write a function that takes a list of words. For each word:
\begin{itemize}
   \item if is shorter then 5 chars, remove the vowels
   \item if is longer or equal to 5 chars, remove the consonants
\end{itemize}

Order the resulting names alphabetically, and then return the concatenated string.

\vspace{10 mm}
Try to be as 'functional' as possible.

\end{frame}

\begin{frame}[fragile]
\frametitle{Solution}
Write a filter that takes a word and:\
\begin{itemize}
   \item if is shorter then 5 chars, remove the vowels
   \item if is longer or equal to 5 chars, remove the consonants
\end{itemize}

\begin{lstlisting}[language=python]
vowels = ['a', 'e', 'i', 'o', 'u']

w = 'anamaria'

filter(lambda c: (len(w)<5)^(c in vowels), w)
# Outputs: aaaia

\end{lstlisting}
\end{frame}

\begin{frame}[fragile]
\frametitle{Solution}
Write a filter that takes a word and:\
\begin{itemize}
   \item if is shorter then 5 chars, remove the vowels
   \item if is longer or equal to 5 chars, remove the consonants
\end{itemize}

\begin{lstlisting}[language=python]
vowels = ['a', 'e', 'i', 'o', 'u']

w = 'asfe'

filter(lambda c: (len(w)<5)^(c in vowels), w)
# Outputs: sf

\end{lstlisting}
\end{frame}

\begin{frame}[fragile]
\frametitle{Solution}
Write a map that takes a list of words. For each word:
\begin{itemize}
   \item if is shorter then 5 chars, remove the vowels
   \item if is longer or equal to 5 chars, remove the consonants
\end{itemize}
\begin{lstlisting}[language=python]
vowels = ['a', 'e', 'i', 'o', 'u']
list_of_words = ['asfe', 'anamaria']

map(
  lambda w: filter(
    lambda c: (len(word)<5)^(c in vowels), w),
  list_of_words)

# Outputs: ['sf', 'aaaia']
\end{lstlisting}
\end{frame}


\begin{frame}[fragile]
\frametitle{Solution}
Now we sort the result of the previous map

\begin{lstlisting}[language=python]
vowels = ['a', 'e', 'i', 'o', 'u']
list_of_words = ['anamaria', 'asfe']

sorted(map(
  lambda w: filter(
    lambda c: (len(w)<5)^(c in vowels), w),
  list_of_words))

# Outputs: ['aaaia', 'sf']
\end{lstlisting}
\end{frame}

\begin{frame}[fragile]
\frametitle{Final Solution}
\begin{lstlisting}[language=python]
vowels = ['a', 'e', 'i', 'o', 'u']
list_of_words = ['anamaria', 'asfe']

reduce(
  lambda x, y: x + y,
  sorted(map(
    lambda w: filter(
      lambda c:(len(w)<5)^(c in vowels), w),
    list_of_words)))

# Outputs: 'aaaiasf'

\end{lstlisting}
\end{frame}

\end{document}

